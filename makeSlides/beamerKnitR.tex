<<'setup', include=FALSE>>=
library(knitr)
### Set knitr compilation options
# knitr::opts_chunk$set(results='hide', echo=FALSE, comment=NA, warning=FALSE, message=FALSE, concordance=TRUE, cache=TRUE) # comment out to print output

# Set smaller font size for chunks (for code display):
knitr::opts_chunk$set(size='footnotesize')

# show R> prompt before R commands
options(prompt='R> ')
knitr::opts_chunk$set(prompt=TRUE) 

# set theme for code output
thm = knit_theme$get("biogoo") # lightgray background
knitr::knit_theme$set(thm)

# Set figure size and resolution
opts_chunk$set(dev=c('pdf','png'), fig.path='figures/color/', fig.align='center', fig.height=c(5, 5), fig.width=c(8, 8), dpi=300, pdf.options(encoding="ISOLatin9.enc")) 

# evaluate fig.cap after a chunk is evaluated
opts_knit$set(eval.after='fig.cap')

# TeX styles in tex document, CSS styles in HTML document
opts_knit$set(self.contained=TRUE)

# modify the default hook to remove spaces before writing the output
hook_output = knit_hooks$get('output')
knit_hooks$set(output=function(x, options) {
  s=options$rm.spaces
  if (!is.null(s)) x=gsub(sprintf('(^|\n)%s', s), '\\1', x)
  hook_output(x, options)
})

### Load R Packages to compile knitr chunks:
# setwd("~/tex/Slides/Lecture-Methods/Research-R/")# temp for convenience
library("ggplot2")      # wide range of plotting capabilities
library("scales")       # labels with %, $ and other units
library("RColorBrewer") # convenient color palettes

### Set ggplot theme for all plots
# change in font sizes may alter legend box positions and/or label overlaps
# White background, legend inside rectangular box, font size 13
# based on theme_bw() whose default values are:
# base_size = 12
# legend.key = theme_rect(colour = "grey80")
# legend.key.size = unit(1.2, "lines")
# legend.text = theme_text(size = base_size * 0.8)
# legend.title = theme_text(size = base_size * 0.8, face = "bold", hjust = 0)
# Remark: ggplot2 has a bug in legend.title vertical spacing when legend.title is NULL, e.g. , legend.title = element_text(vjust = 1)
# to do: add e.g. legend.justification = c(0,0)
theme_slides <- function(x) {
  theme_bw(x) + 
      theme(legend.key = element_blank(), legend.background = element_rect(colour = 'black', fill = 'white'))
}
theme_set(theme_slides(12))

### Set Line, Color & Shape Schemes
# 1. linetype scheme (not extensively tested)
linePalette <- c("solid", "dotted", "dashed", "dotdash", "longdash", "twodash", "13", "31", "1221")# "33" means 3 units on followed by 3 units off
# 2. shape scheme (is there a package for this?)
shapePalette <- c(17, 21, 15, 25, 19, 23, 18, 2, 16, 0, 8)
shapePalettePaired <- c(17, 2, 16, 21, 15, 0, 18, 23, 24)
  ## sp <- shapePalettePaired
  ## ggplot(data.frame(x = 1:length(sp) , y = 1:length(sp), z = sp), aes(x = x, y = y)) + geom_point(aes(shape = z), size = 10) + scale_shape_identity()
# 3. color scheme (with RBrewerColor package)
colorPalette <- colorRampPalette(brewer.pal(9, "Set1"))(9)# extract palette colors
# fill scheme (with RBrewerColor package)
fillPalette <- "Blues"  # shades of blue for filling areas of stack plots
shadePalette <- c("blue", "red", "green")  # light colors for shading areas
### Set a Black & White Scheme
# to set bw scheme, uncomment 4 lines below
     #colorPalette <- rep("#000000", 9)                 # black and white figures
     #fillPalette <- "Greys"                            # black and white figures
     #shadePalette <- c("grey50", "grey90", "grey95")   # grey color for shading
     #opts_chunk$set(fig.path='figures/bw/') # save in sub-directory
@
